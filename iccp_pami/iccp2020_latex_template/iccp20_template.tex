%% iccp20_template.tex
%% Created by Ayan Chakrabarti from the IEEE bare_jrnl_compsoc.tex file.
%%
%% bare_jrnl_compsoc.tex
%% V1.4b
%% 2015/08/26
%% by Michael Shell
%% See:
%% http://www.michaelshell.org/
%% for current contact information.
%%
%% This is a skeleton file demonstrating the use of IEEEtran.cls
%% (requires IEEEtran.cls version 1.8b or later) with an IEEE
%% Computer Society journal paper.
%%
%% Support sites:
%% http://www.michaelshell.org/tex/ieeetran/
%% http://www.ctan.org/pkg/ieeetran
%% and
%% http://www.ieee.org/

%%*************************************************************************
%% Legal Notice:
%% This code is offered as-is without any warranty either expressed or
%% implied; without even the implied warranty of MERCHANTABILITY or
%% FITNESS FOR A PARTICULAR PURPOSE! 
%% User assumes all risk.
%% In no event shall the IEEE or any contributor to this code be liable for
%% any damages or losses, including, but not limited to, incidental,
%% consequential, or any other damages, resulting from the use or misuse
%% of any information contained here.
%%
%% All comments are the opinions of their respective authors and are not
%% necessarily endorsed by the IEEE.
%%
%% This work is distributed under the LaTeX Project Public License (LPPL)
%% ( http://www.latex-project.org/ ) version 1.3, and may be freely used,
%% distributed and modified. A copy of the LPPL, version 1.3, is included
%% in the base LaTeX documentation of all distributions of LaTeX released
%% 2003/12/01 or later.
%% Retain all contribution notices and credits.
%% ** Modified files should be clearly indicated as such, including  **
%% ** renaming them and changing author support contact information. **
%%*************************************************************************


\documentclass[10pt,journal,compsoc]{IEEEtran}
\newif\ifpeerreview

%%% Important: for camera ready submissions, replace the following line
%%% with \peerreviewfalse
\peerreviewtrue


\usepackage[nocompress]{cite}
\usepackage{url}
\usepackage{amsmath,amssymb,graphicx}

\usepackage{lipsum} % Only used to generate random text.


\usepackage[switch]{lineno}

% Insert your paper ID and information below
\newcommand{\paperID}{XXXX}

% Enter your paper title below
\title{ICCP 2020 Paper Template}

% Enter your author information before
% Note this is only necessary for the camera review. Submissions are anonymized.
\author{Michael~Shell,~\IEEEmembership{Member,~IEEE,}
        and~Jane~Doe,~\IEEEmembership{Life~Fellow,~IEEE}% <-this % stops a space
\IEEEcompsocitemizethanks{\IEEEcompsocthanksitem M. Shell is with the Department
of Electrical and Computer Engineering, Georgia Institute of Technology, Atlanta,
GA, 30332.\protect\\
% note need leading \protect in front of \\ to get a newline within \thanks.
E-mail: see http://www.michaelshell.org/contact.html
\IEEEcompsocthanksitem J. Doe is with Anonymous University.}% <-this % stops an unwanted space
}


\begin{document}

\IEEEtitleabstractindextext{%
\begin{abstract}
The abstract goes here. \lipsum[1]
\end{abstract}

\begin{IEEEkeywords} % Enter keywords here
Computational Photography
\end{IEEEkeywords}
}


% Make Title
\ifpeerreview
\linenumbers \linenumbersep 15pt\relax 
\author{Paper ID \paperID\IEEEcompsocitemizethanks{\IEEEcompsocthanksitem This paper is under review for ICCP 2020 and the PAMI special issue on computational photography. Do not distribute.}}
\markboth{Anonymous ICCP 2020 submission ID \paperID}%
{}
\fi
\maketitle



% The first section title should be wrapped inside a \IEEEraisesectionheading as follows.
\IEEEraisesectionheading{
  \section{Introduction}\label{sec:introduction}
}
% The very first letter of the paper is a 2 line initial drop letter
% followed by the rest of the first word in caps.
% 
% form to use if the first word consists of a single letter:
% \IEEEPARstart{A}{demo} file is ....
% 
% form to use if you need the single drop letter followed by
% normal text (unknown if ever used by the IEEE):
% \IEEEPARstart{A}{}demo file is ....
% 
% Some journals put the first two words in caps:
% \IEEEPARstart{T}{his demo} file is ....
% 
% Here we have the typical use of a "T" for an initial drop letter
% and "HIS" in caps to complete the first word.
\IEEEPARstart{T}{his} demo file is intended to serve as a ``starter
file'' for ICCP 2020 submissions produced under
\LaTeX~\cite{kopka-latex} using IEEEtran.cls version 1.8b and later.

For submissions for review, the paper uses the lineno package which
might require you to compile under latex twice to get the line numbers
to align correctly.

\lipsum[2-4]


\begin{figure}[!t]
\centering
\framebox[\columnwidth]{\parbox{0.9\columnwidth}{~\\~\\~\\~\\~\\}}
\caption{Example one-column figure.}
\end{figure}


\section{Related Work}
\lipsum[5-6]

\begin{figure*}[!t]
\centering
\framebox[\textwidth]{\parbox{0.9\textwidth}{~\\~\\~\\~\\~\\~\\~\\}}
\caption{Example two-column figure.}
\end{figure*}

\section{Proposed Method}
\lipsum[7]

\subsection{Subsection Heading Here}
\lipsum[8-9]

\subsubsection{Subsubsection Heading Here}
\lipsum[10-12]


\section{Experimental Results}

\lipsum[13-17]

\begin{table}[!t]
\renewcommand{\arraystretch}{1.3}
\caption{An Example of a Table}
\centering
\begin{tabular}{c||c|c|c}
\hline
One & Two & One & Two\\
\hline\hline
Three & Four &Three & Four\\
\hline
Three & Four &Three & Four\\
\hline
Three & Four &Three & Four\\
\hline
\end{tabular}
\end{table}


\section{Conclusion}
\lipsum[18]



% Any acknowledgments to only be included in camera ready
\ifpeerreview \else
\section*{Acknowledgments}
The authors would like to thank...
\fi

\bibliographystyle{IEEEtran}
\bibliography{references}



\ifpeerreview \else
%%%% For the camera ready version, please fill out this
%%%% biography. Your camera ready should be within a 12 page limit
%%%% including acknowledgments, references and biography.

% If you have an EPS/PDF photo (graphicx package needed) extra braces are
% needed around the contents of the optional argument to biography to prevent
% the LaTeX parser from getting confused when it sees the complicated
% \includegraphics command within an optional argument. (You could
% create your own custom macro containing the \includegraphics command
% to make things simpler here.)
% \begin{IEEEbiography}[{\includegraphics[width=1in,height=1.25in,clip,keepaspectratio]{mshell}}]{Michael Shell}
% or if you just want to reserve a space for a photo:

\begin{IEEEbiography}{Michael Shell}
Biography text here.
\end{IEEEbiography}

% insert where needed to balance the two columns on the last page with
% biographies
%\newpage

% if you will not have a photo at all:
\begin{IEEEbiographynophoto}{John Doe}
Biography text here.
\end{IEEEbiographynophoto}

% You can push biographies down or up by placing
% a \vfill before or after them. The appropriate
% use of \vfill depends on what kind of text is
% on the last page and whether or not the columns
% are being equalized.
%\vfill

\fi

\end{document}


