\documentclass[12pt]{report}
\begin{document}

And so it begins

\chapter{Introduction}

\section{Thesis Roadmap}
\section{Light field background}

why don't traditional methods work with light fields?
Why don't methods adapted for typical light fields work with our light fields?

\chapter{Robotic Collection and Rendering of Light Fields}
\section{methods for collection}
Gauging what robots are sufficient, and if robots aren't sufficient, how do slam methods work if I was to just move the robot arm around?
Graphs should be floating around somewhere for this.

Review your previous years reviews as well for this earlier stuff.
\section{methods for rendering}
\subsection{traditional methods}
synthetic aperture dependent on parameterizations
drawbacks

\section{focused plenoptic cameras}
moving on to different types of plenoptic cameras

Outline the different ways of rendering, and do it mathematically in terms of how you dehex and 1D interpolate to get sais

Then show how this is unnecessary for focused cameras.
Possible to show aperture abilities/refocusing based on depth? that'd be a nice little addition.

Don't forget to pull from 3d images from focused plenoptic cameras and their versions of pulling data.

Don't forget your structure from motion.

\chapter{Low Light Light Field}
\section{A Low-Light Dataset}

\section{Looking at low light}

\chapter{Depth from Low Light}
\section{creating the depth network and a back prop method for light fields}
Showing the movement of lenses

How does the cost volume work and what is it we propose to learn?

What is the problem with cost volumes?

How do you deal with very low resolution images?

What's the deal with network structure in the sense of starting large and compressing down for features?
\section{Video}
Struggles of capturing a video dataset for low light enhancement

At some point, perhaps outline what makes for good depth maps.

\subsection{Temporal Noise and Exposure Variation}
\subsection{Learning Exposure Values}

\chapter{Burst low light}

\chapter{Conclusion}

\end{document}