\documentclass[12pt]{report}
\usepackage[utf8]{inputenc}
\usepackage{graphicx}
\usepackage[tmargin=2cm, lmargin=4cm, rmargin=2.5cm, bmargin=4cm, paperwidth=8.267in, paperheight=11.692in]{geometry}
\usepackage{amsfonts}
\usepackage{array}
\usepackage{indentfirst}
\graphicspath{ {images/} }
\usepackage{lipsum}
\usepackage{verbatim}
\usepackage{titlepic}
\usepackage{amsmath}
%\usepackage{titlesec}

\begin{document}

\title{Swarm-Collected Light Field for 3D Object Recreation  \vspace{2.5cm}}	%\includegraphics[scale=0.2]{university_edinburgh.jpg}

\date{
	\centering
	20 January 2017
}

\maketitle
\section*{Project Aim}
Utilize multiple drones to quickly and autonomously collect images of an area or object of interest via adaptively sampling and pre-built image pipeline. The resulting data should be processed to allow for occlusion removal, novel perspective creation, and 3D reconstruction.
 
\section*{Function}
This project would allow for new ways to visualize areas of interest. From a tactical perspective, the light field for an area may be collected, allowing for an immersive analysis of said area. A drone swarm is well equipped to gather this data quickly and efficiently, with the potential to adapt to changes in number or environment. 

\section*{Criterion}
\begin{itemize}
\item Drones are equipped with cameras and capable of collecting images.
\item Localization method such as ORB-SLAM2 used to determine drone's location relative to area of interest.
\item The area is adaptively sampled.
\item Resulting data is piped through pre-built light field image interpolation and rendering pipeline. 
\item User is able to control view, focus, and occlusion factors.
\end{itemize}

\end{document}